%%%%%%%%%%%%%%%%%%%%%%%%%%%%%%%%%%%%%%%%%%%%%%%%%%%%%%%%%%%%%%%%%%%%%%%%%%%%%%%%
%% 
\cleardoubleoddpage%  Make sure to start each chapter on a new odd page
\chapter{Introduction}
Respiratory diseases are a leading cause of premature mortality worldwide. With over four million annual deaths attributed to these diseases, early identification and treatment efforts are imperative~\cite{ferkol2014global}. The use of chest auscultation, a technique in which respiratory sounds are analyzed with instruments like stethoscopes, is a simple and effective way for diagnosing respiratory diseases. \\
Automated systems for detecting sound anomalies have become of increasing relevance in the medical field and are driving machine learning research~\cite{bohadana2014fundamentals}. They have the potential to improve diagnostic accuracy for healthcare professionals and provide initial assessments for patients, ultimately leading to more efficient allocation of healthcare resources. 
% Stethoscope got digital -> automation possible
% Write about how to spot anomalies in the sounds and how it can be automated
% Why it is of benefit to use only healthy sounds for training: The importance of semi-supervised learning
% Mention semi-supervised so I can talk abo∏ut it in preprocessing

\section{Motivation}



\section{Objectives and Approach}

% Introduce dataset shortly
% Introduce methods shortly
% Objectives

% should showcase existing anomaly detection approaches to respiratory sound analysis and give an outlook which methods might work better and which ones to pursue trying out

\section{Outline}



%% 
%%%%%%%%%%%%%%%%%%%%%%%%%%%%%%%%%%%%%%%%%%%%%%%%%%%%%%%%%%%%%%%%%%%%%%%%%%%%%%%%
